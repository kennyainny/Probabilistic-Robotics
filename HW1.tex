\documentclass[letterpaper]{article}
\title{CS8803: STR, Spring 2018: Problem Set 1}
\date{Due: Wednesday, January 24th, beginning of the class}
\usepackage[margin=1in]{geometry}
\usepackage{amsmath}
\usepackage{amssymb}
\usepackage{url}
\usepackage{graphicx}
\usepackage{color}

\begin{document}

\maketitle

%%%%%%%%%%%%%%%%%%%%%%%%%%%%%%%%%%%%%%%%%%%%%%%%%%%%%%%%%%%%%%%%%%%%
% Instructions
%%%%%%%%%%%%%%%%%%%%%%%%%%%%%%%%%%%%%%%%%%%%%%%%%%%%%%%%%%%%%%%%%%%%

\subsubsection*{Instructions}  There are 2 questions on this assignment (3 pages).  These are short, simple problems. The maximum answer length should not exceed the space provided.

%%%%%%%%%%%%%%%%%%%%%%%%%%%%%%%%%%%%%%%%%%%%%%%%%%%%%%%%%%%%%%%%%%%%
% Problem 1
%%%%%%%%%%%%%%%%%%%%%%%%%%%%%%%%%%%%%%%%%%%%%%%%%%%%%%%%%%%%%%%%%%%%

\section{Markov Assumption}
\emph{This is not meant to be a tricky question, just one to get you thinking about an important assumption that is often made. One paragraph each, at most.}


\subsection{}
Give one robotic example where the Markov assumption is used (correctly or not).  Is the assumption valid or not?  Explain.

\vspace{5mm}

The Markov assumptions is used in robot localization given a fixed map of the robots environment. In the case of robot localization it's safe to use this assumption under certain circumstances, such as a static environment, the use of sensors with minimal measurment error, and and a accurate map of the robots environment.  If the environment was dynamic then the Markov assumption would be violated.


\subsection{}
Give one real-world example where the Markov assumption is used (correctly or not).  Is the assumption valid or not?  Explain.
\vfill

\newpage

%%%%%%%%%%%%%%%%%%%%%%%%%%%%%%%%%%%%%%%%%%%%%%%%%%%%%%%%%%%%%%%%%%%%
% Problem 2
%%%%%%%%%%%%%%%%%%%%%%%%%%%%%%%%%%%%%%%%%%%%%%%%%%%%%%%%%%%%%%%%%%%%
\section{Bayes Filter Derivation}
Recall the derivation for the Bayes Filter in the slides:
\begin{eqnarray}
Bel\left( x_t \right) & = & P\left( x_t | u_{1:t}, z_{1:t} \right) \\
& \ldots & \\
\label{before} & \propto &  P\left( z_t | x_t \right) \int P\left( x_t | u_t, x_{t-1} \right) P\left( x_{t-1} | u_{1:t}, z_{1:t-1} \right) dx_{t-1} \\
\label{after}  & \propto &  P\left( z_t | x_t \right) \int P\left( x_t | u_t, x_{t-1} \right) P\left( x_{t-1} | u_{1:t-1}, z_{1:t-1} \right) dx_{t-1} \\
   & \propto &  P\left( z_t | x_t \right) \int P\left( x_t | u_t, x_{t-1} \right) Bel\left(x_{t-1} \right) dx_{t-1}
\end{eqnarray}

\subsection{}
In the slides, the Markov assumption is invoked between lines~\eqref{before} and~\eqref{after} to drop $u_t$.  Why is this incorrect? (Why does the Markov assumption not enable you to drop $u_t$)
\vfill

\subsection{}
Provide a counter-example where knowing $u_t$ gives you information about the state $x_{t-1}$.

\vspace{5mm}
Any system that is observable uses the current value of the input and output to determine past values of the state. 

\newpage
\subsection{}
What does the book assume about the controls $u$ in order to drop $u_t$ from the derivation of the Bayes Filter?  Is this assumption reasonable?  Why or why not?
\vspace{5mm}

The book assumes that the state is complete, which means that the no variables prior to the current state that will influence the stochastic evolution of future states.  This is similar to the Markov assumption therfore the assumption isn't reasonable in the real world where the environment may be changing or there may be large errors in the robots sensor readings. 

\subsection{}
There are weaker assumptions you can make about the controls to still drop $u_t$ from the Bayes Filter derivation.  Think conditional independence, and derive how $u_t$ is dropped between lines~\eqref{before} and~\eqref{after}.

\vspace{150pt}
\vfill

\end{document}
